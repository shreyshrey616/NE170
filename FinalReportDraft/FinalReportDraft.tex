
\documentclass[]{article}
\usepackage{mathrsfs}
\usepackage{amsmath}
\usepackage{graphicx}
\usepackage[top=1in,right=1in,left=1in,bottom=1in]{geometry}
\usepackage{setspace}
\usepackage{indentfirst}
\usepackage[margin=10pt,labelfont=bf,justification=raggedright,singlelinecheck=false]{caption}
\usepackage{color}
\usepackage{float}

\usepackage[pdftex]{hyperref}
\hypersetup{
       colorlinks=true,
       linkcolor=blue,
       citecolor=blue
}

\setlength{\abovecaptionskip}{6pt}   
\setlength{\belowcaptionskip}{6pt}   

% Begin
\begin{document}


%%%%%%%%%%%         PARAGRAPH SPACING       %%%%%%%%%%%%%%%%%%%%
% basically \doublespacing and \singlespacing. You can include them at any point in the document to change it for
% that point onward.
\doublespacing
\interfootnotelinepenalty=10000

% Set up a titlepage
\begin{titlepage}
\null\vspace{\stretch{1}}
\begin{center}
\textsc{\LARGE Environmental Sample Analysis by Means of Neutron Activation Analysis}\\[1.5cm]

\textsc{\Large NE 170 Project Report}\\[0.5cm]
\textsc{\Large Eduardo Zagal, Peter Thomas, Shreyas Srinivasan, Zach Levine}\\[0.5cm] % Authors here
\textsc{\Large Advisers: Keenan Thomas and Professor Kai Vetter}\\[0.5cm]
\textsc{\Large May 12, 2016}\\[0.5cm] % Date that you turn it in
\end{center}
\vspace{\stretch{1}}
\end{titlepage}
\pagebreak


%%%%%%%%%%%%          TABLE OF CONTENTS          %%%%%%%%%%%%%%%%%%%%%%
\tableofcontents

%%%%%%%%%%%%%             PAGE BREAK            %%%%%%%%%%%%%%%%%
\pagebreak


%%%%%%%%%%%%          SECTION HEADINGS          %%%%%%%%%%%%%%%%%%%%%%
% To make section headings, use \section{Section Title}, adding a * makes a section title without numbering it.
% subsections are \subsection{Title} and \subsubsection{Title} and so on
\singlespacing

\section{Introduction}
Our team aims to determine the elemental composition, specifically trace metals, of kelp, fish, and other sea life along the west coast of the Americas, particularly in the San Francisco Bay and Long Beach, though we will be studying samples from Alaska all the way to Chile. In particular, we would like to see if there are any major differences in isotopes found in samples from different parts of the West Coast, and whether the isotopes found in the different samples will present a toxicity or radiological health hazard to the peoples living along the Pacific.

We will be using neutron activation analysis (NAA) to determine the elemental composition of our samples. NAA involves irradiating test samples with a high neutron flux in order to activate the isotopes present in the sample. These isotopes then decay, emitting characteristic gamma rays that allow us to identify the isotopes that were originally in the sample. Based on the half-life of the isotope, its branching ratio at a particular gamma energy, the counts registered on the detector, and detector efficiency, we can also determine the amount of the isotope in the test sample. NAA provides advantages that other techniques do not. NAA is relatively unobtrusive, and will not cause significant damage to the test sample. As neutrons can penetrate deeply into a material, NAA can be used on bulk samples. These samples do not need the careful preparation that is needed for other techniques to determine elemental composition. Thus, NAA provides a fairly accurate and precise measurement of elemental compositions without causing physical damage to the sample, and without the hassle of extended preparation time. With these advantages come some significant disadvantages that we will have to work around. One is that materials activated by neutrons remain radioactive for a long time after irradiation takes place. Depending on the isotopes in the sample, and the half-lives of those isotopes, this can make working with activated materials hazardous. This should not be an overwhelming concern for our team, as the samples that we will be working with are not expected to contain hazardous isotopes in the concentration needed to be harmful to us. Nevertheless, presence of toxic and radiologically active materials should be noted. The other concern is that there are only a select few facilities that have the capabilities needed to perform neutron activation on test samples. Fortunately, the McClellan Nuclear Research Center (MNRC) is relatively close and is able to perform neutron activation with the Training, Research, and Isotope Production General Atomics (TRIGA) reactor, an advantage that will be crucial to the success of our project.

            
\pagebreak


\section{Goals}

Our ultimate goal in this project is to be able to communicate to the public radiological information about things they regularly interact with.  Since the Fukushima Daiichi Power incident in 2011, public concern about all things nuclear has dramatically increased.  The air people breathe, the fish they eat, and the bodies of water that surround them can be analyzed radiologically.  Publishing the analysis for people to see creates a more radiologically informed public.  We intend to accomplish this overarching goal by meeting these more specific needs:
\begin{itemize}
\item Analyze recently taken NAA data on kelp from all over the West Coast
\item Obtain new samples from the bay, carry out NAA on them at MNRC, and analyze those results
\item Perform NAA measurements on seaweed samples
\item Find new samples to perform NAA and gamma-ray analysis on
\item Compile results and write an article on the techniques used and findings
\item Take important data and articles and publish them to the Radwatch website
\end{itemize}

\section{Roles}

\textbf{Shreyas}:
\begin{itemize}
\item Initial data analysis
\item Maintain living document(s) as necessary 
\item Make presentation(s) as necessary
\item Compile group work into LATEX
\item Lead in publishing to Radwatch site
\end{itemize}

\textbf{Zach}: 
\begin{itemize}
\item Initial data analysis
\item Lead in communicating with instructor on a weekly basis on behalf of the group
\item Lead in communicating with points of contact on behalf of the group
\item Generate efficiency curve(s) for detectors being used
\item Help with publishing to Radwatch site
\end{itemize}

\textbf{Eduardo}:
\begin{itemize}
\item Initial data analysis
\item Find new, useful samples for more analysis
\item Help with new sample data analysis
\item Lead in writing article(s) to be published
\item Help with publishing to Radwatch site
\end{itemize}

\textbf{Peter}:
\begin{itemize}
\item Initial data analysis
\item Find new, useful samples for more analysis
\item Lead in data analysis for new samples tested
\item Help with writing article(s)
\item Help with publishing to Radwatch site
\end{itemize}

\pagebreak

\section{Timeline}

Figure 1 shows a Gantt chart timeline of the various tasks required for this project. The first task revolves around the existing NAA data. Following the procedure described in the introduction, we must identify the composition of the existing samples. For the most part, the timeline was followed to the best of our abilities. Unfortunately, due to logistics with irradiation at MNRC and receiving the samples, we were unable to analyze the short run data and some of the samples from the second irradiation. Before publishing to the website, the short run data and all second irradiation samples will be analyzed. 

\begin{figure}[h]
\centering
\includegraphics[scale=0.4]{GanttChart}
\caption{This figure shows a Gantt chart with the expected timeline to finish the various tasks for this project.}
\end{figure} 

\pagebreak

\section{Critical Equipment}
We will be using the following instruments/tools:


\begin{itemize}
\item High Purity Germanium (HPGe) Detector in 1110C
\item PeakEasy 4.81
\item MNRC reactor
\item High precision scale
\item Plastic vials for sample preparation



\end{itemize}

\subsection{Detector Efficiency Curve}

Figure 2 shows the detector effiency curve for the HPGe detector used in room 1110C. The sources used to generate the efficiency curve were $^{137}$Cs, $^{60}$Co, $^{152}$Eu, $^{22}$Na, and $^{228}$Th. We used a plethora of sources in order to cover a wide range of gamma ray energies and find the efficiency over the entire energy spectrum we will be measuring. Using this efficiency curve, we will be able to determine the inital elemental concentrations of our samples.


\begin{figure}[h]
\centering
\includegraphics[scale=0.6]{Efficiency}
\caption{This figure shows the dectector efficiency curve for the HPGe detector in room 1110C.}
\end{figure} 

\pagebreak
 

\section{Irradiation Procedure}

\subsection{Sample Preparation}
Before preparing our samples, we decided to determine what new samples to activate. We decided to choose various samples of fish, seaweed, and seashells for analysis. The following list shows the elements expected to be seen in the different samples.

\textbf{Seaweed}:
\begin{itemize}
\item Potassium, Sodium, Calcium, Magnesium and Sulfur
\item Trace amounts of Zinc, Copper, Chlorine, Cobalt, Manganese, Selenium, Bromine, Iron, and Arsenic
\end{itemize}

\textbf{Seashells}:
\begin{itemize}
\item Calcium, Carbon, Oxygen
\end{itemize}

\textbf{Fish}:
\begin{itemize}
\item Carbon, Oxygen, Hydrogen, Sulfur, and Nitrogen
\item Trace amounts of various metals, such as Sodium, Aluminum, and Iron
\item Looking for Mercury-203 at 279 keV
\end{itemize}


Preparation of the samples for irradiation required two individuals, both wearing latex gloves. One individual handled the weighing of the vial using the high precision scale before and after a sample was placed inside. The other individual used tongs to carefully place each sample into the vial, making sure to not contaminate the outside of the vial to prevent radioactive contamination from forming during irradiation. After all the samples were placed in vials, the vials were soldered shut to ensure that the vials would not open inside the reactor. The vials were then transported to MNRC where the irradiation occurred. 

\subsection{Long Run Procedure}

For the long run, we used samples of fish, seaweed, and seashells in order to diversify our sample data and findings.  Again, after receiving approval from MNRC to perform long run irradiation on our samples, we took the samples to MNRC.  The samples were placed in the neutron transmutation doping (NTD) position, which has a lower flux, but allows for irradiation at a power level of 1 MW for many hours, as opposed to the pneumatic transfer system (PTS). Figure 3 illustrates the locations of the PTS and NTD in the reactor core.  After waiting a few days for the samples to decay to safe levels, we received the samples in Etcheverry and counted them using the HPGe detector in the 1110C laboratory.

\subsection{Short Run Procedure}
For the short run, we used four samples from the initial set of kelp samples.  After receiving clearance from MNRC to run these samples for a short-lived isotope test, we brought the samples to MNRC.  We put our sample vials inside rabbit tubes, which eventually went in the PTS. We ran the reactor at a power level of 20 kW, for a total of 100 seconds of irradiation. Immediately after the samples decayed to a level safe to handle, MNRC ran counts on the samples up to an hour in length.  

\begin{figure}[htb!]
\centering
\includegraphics[scale=0.5]{Reactor}
\caption{This figure shows the neutron transmutation doping (NTD) and pneumatic transfer system (PTS) positions inside the reactor core of the MNRC TRIGA reactor.}
\end{figure} 
\pagebreak



\section{Spectra Analysis}
\subsection{Kelp Long Run Spectra}
Figure 4 shows a sample kelp spectrum that we analyzed. Our initial analysis found the characteristic gamma rays of ${82}$Br and ${24}$Na. This shows that bromine and sodium were most likely present in our original sample, and using the neutron flux, neutron capture cross section, counts in peak, and detector efficiency, we can determine the initial composition of bromine and sodium in our sample.

\begin{figure}[htb!]
\centering
\includegraphics[scale=0.5]{ExampleSpectra}
\caption{This figure shows a NAA spectrum for the kelp sample from Sitka, AL. $^{82}$Br and $^{24}$Na gamma rays are very prominent in the spectra.}
\end{figure} 
\pagebreak

\subsection{Kelp Short Run Spectra}
\begin{figure}[htb!]
\centering
\includegraphics[scale=0.18]{SampleB}
\caption{This figure shows the short run NAA spectrum for the kelp sample from Sitka, AL. In addition to $^{82}$Br and $^{24}$Na gamma rays, $^{38}$Cl, $^{56}$Mn, and $^{126}$I were seen, among other elements.}
\end{figure} 

\pagebreak

\subsection{Fish Long Run Spectra}

\begin{figure}[htb!]
\centering
\includegraphics[scale=0.7]{FishSpectra}
\caption{This figure shows an example NAA spectrum for the fish used in the second long run analysis. Many different gamma rays can be seen in the spectra, including $^{82}$Br, $^{24}$Na, $^{203}$Hg, and $^{134}$Cs.}
\end{figure} 



\subsection{Seaweed Long Run Spectra}


\begin{figure}[htb!]
\centering
\includegraphics[scale=0.8]{SeaweedSpectra}
\caption{This figure shows an example NAA spectrum for the seaweed used in the second long run analysis. Many different gamma rays can be seen in the spectra, including $^{82}$Br, $^{24}$Na, $^{76}$As, and $^{65}$Zn.}
\end{figure} 

\pagebreak

\subsection{Seashell Long Run Spectra}

\begin{figure}[htb!]
\centering
\includegraphics[scale=0.8]{SeashellSpectra}
\caption{This figure shows an example NAA spectrum for the seashells used in the second long run analysis. Many different trace metals can be seen, most prominently $^{47}$Ca, and $^{47}$Sc.}
\end{figure} 



\pagebreak
\section{Data Analysis and Results}
\subsection{Concentration Calculation}

\begin{figure}[h]
\centering
\fbox{\includegraphics[scale=0.5]{Eq3}}
\caption{This figure shows the orthodox equation used to determine the relative abundance a given element in an unknown sample. Dependencies include irradiation flux, activity, cross section, time of irradiation, natural isotopic abundance, detector efficiency, and gamma ray branching ratio.}
\end{figure}

The conventional way to determine relative elemental composition in an unknown sample is given by the equation in Figure 9. However, by using standard pottery, we are able to determine the elemental composition of an unknown sample without detector efficiency, irradiation flux, and cross section. Standard pottery is clay pottery which contains a wide range of chemical elements. Using NAA, we can activate the pottery and use it as a chemical fingerprint for many different elements. Many minor components of the pottery are activated via NAA and these component compositions are known to high accuracy, making the standard pottery a valuable tool in determining the composition of an unknown sample. Figure 10 shows the elemental composition of the standard pottery used as part of this project.

\begin{figure}[h]
\centering
\fbox{\includegraphics[scale=0.3]{stdpottery}}
\caption{This figure shows the elemental compositions in the standard pottery. These compositions were used in the calculation of the sample compositions.}
\end{figure}

The counts per second per gram, or $\frac{cps}{g}$ for a given sample is given by $$\frac{cps}{g} = \frac{N\Phi\sigma(1-e^{-\lambda t})\epsilon \beta_{ \gamma}}{m * LT}$$ where N is number of atoms, $\Phi$ is irradiation flux, $\sigma$ is cross section, $\epsilon$ is detector efficiency, $\beta_{\gamma}$ is branching ratio, LT is live time, and m is sample mass. We can write the activity A of a given sample as $$A=N\Phi\sigma(1-e^{-\lambda t})$$, allowing us to rewrite $\frac{cps}{g}$ of a given sample as $$\frac{cps}{g} = \frac{A\epsilon\beta_{\gamma}}{m*LT}$$ Using a simple ratio between level of sample activation and elemental concentration $\frac{A}{C}$, we can say that $$\frac{\frac{cps}{g}_{sample}|_{t=0}}{C_{sample}} = \frac{\frac{cps}{g}_{pottery}|_{t=0}}{C_{pottery}}$$ Using this ratio, we can calculate the elemental concentration in a given sample with the standard pottery. $$C_{sample} = \frac{\frac{cps}{g}_{sample}|_{t=0} C_{pottery}}{\frac{cps}{g}_{pottery}|_{t=0}}$$

The error associated ratio method for calculating sample concentration can be calculated using the error propagation formula $$\sigma_F^2 = \sum_{i=1}^N (\frac{dF}{dx_i})^2 \sigma_{x_i}^2$$. Expanding this out for the formula given above, we get that the error in concentration can be given by $$\sigma_{C_{sample}}^2 = (\frac{C_{pottery}}{\frac{cps}{g}_{pottery}|_{t=0}})^2\sigma_{\frac{cps}{g}_{sample}}^2 +  (\frac{\frac{cps}{g}_{sample} C_{pottery}}{(\frac{cps}{g}_{pottery})^2})^2|_{t=0}\sigma_{\frac{cps}{g}_{pottery}}^2 +  (\frac{\frac{cps}{g}_{sample}|_{t=0}}{\frac{cps}{g}_{pottery}|_{t=0}})^2\sigma_{C_{pottery}}^2$$

In the event that a characteristic gamma ray peak was not seen in any of the spectra for a given sample, an upper bound was given for the composition. By analyzing the energy region where the peak should be (for example, 1525keV for $^{42}$K), we can determine the error in peak area by taking the square root of counts in the energy reason. By using this as the number of counts, we can propogate a 1$\sigma$ error bound through the composition calculation, giving an upper limit on the elemental conentration. 

\subsection{Kelp Sample Long Run Results}

Table 1 shows the elemental compositions for trace metals of the various kelp samples from the initial long run analysis. 

\begin{table}[h]
\centering
\caption{ This table shows various elemental composition percentages of metals for the initial kelp NAA and error. The composition of Scandium is shown in PPM due its small composition of the samples.}
\resizebox{\columnwidth}{!}{
\begin{tabular}{ | l | l | l | l | l | l | l | l | l | l | }
\hline
	Sample Number & Kelp ID & Sample Location & \% K & \% Na  & \% As & \% Rb & \% Zn & \% Fe & ppm Sc \\ \hline
	1 & Sitka-1 & Sitka, AK & 7.3 $\pm$ 0.3 & 4.70 $\pm$ 0.08 & 0.0096 $\pm$ 0.0007 & 0.0045 $\pm$ 0.0004 & 0.0057 $\pm$ 0.0005 & 0.0181 $\pm$ 0.0006 & 2.69 $\pm$ 0.07 \\ \hline
	2 & RES-1 & Tofino, BC & 2.8 $\pm$ 0.1 & 4.07 $\pm$ 0.07 & $<$0.00165 & 0.0044 $\pm$ 0.0004 & $<$ 0.0013 & $<$0.0042 & 5.4 $\pm$ 0.3 \\ \hline
	3 & BML-1 & Van Damme State Park, CA & 3.8 $\pm$ 0.5 & 3.69 $\pm$ 0.07 & 0.007 $\pm$ 0.0006 & 0.0043 $\pm$ 0.0005 & 0.0034 $\pm$ 0.0001 & 0.0580 $\pm$ 0.0006 & 0.17 $\pm$ 0.01 \\ \hline
	4 & UCSCTER-1 & Santa Cruz, CA & $<$2.1 & 3.95 $\pm$ 0.06 & 0.0083 $\pm$ 0.0006 & 0.0046 $\pm$ 0.0004 & 0.0043 $\pm$ 0.0003 & 0.0156 $\pm$ 0.0003 & 7.3 $\pm$ 0.1 \\ \hline
	5 & CAM-1a & Cambria, San Luis Obispo County, CA & 3.4 $\pm$ 0.5 & 4.09 $\pm$ 0.06 & 0.0090 $\pm$ 0.0007 & 0.0041 $\pm$ 0.0004 & 0.0040 $\pm$ 0.003 & 0.0150 $\pm$ 0.0004 & 6.4 $\pm$ 0.4 \\ \hline
	6 & REED-1c & Santa Barbara, CA & 3.7 $\pm$ 0.6 & 4.26 $\pm$ 0.06 & 0.0106 $\pm$ 0.0008 & 0.0052 $\pm$ 0.0005 & 0.0047 $\pm$ 0.0004 & 0.0271 $\pm$ 0.0006 & 7.8 $\pm$ 0.1 \\ \hline
	7 & Oxy-1 & Rancho Palos Verdes, Los Angeles County, CA & 6.8 $\pm$ 0.6 & 3.22 $\pm$ 0.06 & 0.0081 $\pm$ 0.0007 & 0.0029 $\pm$ 0.0003 & 0.0039 $\pm$ 0.0003 & 0.0058 $\pm$ 0.0001 & 1.73 $\pm$ 0.03 \\ \hline
	8 & LB-1 & Long Beach, Los Angeles County, CA & 3.2 $\pm$ 0.5 & 5.21 $\pm$ 0.08 & 0.0095 $\pm$ 0.0009 & 0.0040 $\pm$ 0.0004 & 0.024 $\pm$ 0.002 & $<$0.0099 & 5.21 $\pm$ 0.2 \\ \hline
	9 & OI-1 & San Clemente, Orange County, CA & 2.7 $\pm$ 0.1 & 2.46 $\pm$ 0.04 & 0.0100 $\pm$ 0.0007 & 0.0039 $\pm$ 0.0004 & 0.0029 $\pm$ 0.0003 & $<$0.0064 & 5.03 $\pm$ 0.2 \\ \hline
	10 & PTL-1 & Point Loma (San Diego), CA & 6.5 $\pm$ 0.7 & 3.10 $\pm$ 0.06 & 0.0084 $\pm$ 0.0007 & 0.0047 $\pm$ 0.0006 & $<$ 0.016  & 0.067 $\pm$ 0.006 & 14.9 $\pm$ 0.9 \\ \hline
	11 & UABC-1 & Baja California, Mexico & 2.5 $\pm$ 0.4 & 3.69 $\pm$ 0.06 & 0.0088 $\pm$ 0.0006 & 0.0046 $\pm$ 0.0004 & 0.0090 $\pm$ 0.0008 & 0.015 $\pm$ 0.001 & 2.4 $\pm$ 0.2 \\ \hline
	12 & UCN-1 & Chile & 6 $\pm$ 1 & 4.23 $\pm$ 0.08 & 0.0045 $\pm$ 0.0004 & 0.0049 $\pm$ 0.0005 & 0.0005 $\pm$ 0.0001 & $<$0.019 & 2.3 $\pm$ 0.4 \\ \hline
	13 & LB-2B & Long Beach, Los Angeles County, CA & 5.4 $\pm$ 0.2 & 4.10 $\pm$ 0.07 & 0.0070 $\pm$ 0.0005 & 0.0018 $\pm$ 0.0002 & $<$ 0.0028 & 0.031 $\pm$ 0.001 & 1.82 $\pm$ 0.05 \\ \hline
	14 & NS6 & Location unknown & 3.3 $\pm$ 0.5 & 5.39 $\pm$ 0.09 & 0.0086 $\pm$ 0.0007 & 0.0049 $\pm$ 0.0005 & 0.0026 $\pm$ 0.0004 & 0.010 $\pm$ 0.002 & 2.5 $\pm$ 0.3 \\ \hline
	15 & NS1 & Location unknown & $<$5.1 & 3.03 $\pm$ 0.05 & $<$ 0.0010 & 0.0016 $\pm$ 0.0001 & 0.0038 $\pm$ 0.0003 & 0.0030 $\pm$ 0.0002 & 1.58 $\pm$ 0.04 \\ \hline
\end{tabular}


}
\end{table}


\subsection{Kelp Sample Short Run Results}
Unfortunately there was a miscommunication with MNRC regarding detector efficiency and binary data. Since we did not take a spectra of the standard pottery on the detectors used at MNRC, we cannot use the ratio method discussed earlier in this report due to the different detector efficiencies. Therefore, we need the efficiency data for the MNRC detectors to calculate the elemental concentration for the short run samples. This was not communicated properly to MNRC and unfortunately we were unable to retrieve the efficiency data prior to the submission of this report. Prior to publishing on the website, the short run data will be analyzed, either by us or students working with RadWatch in the near future.

\subsection{Fish Long Run Results}

Due to logistical issues with receiving the samples, we were unable to start counting the samples for approximately 10 days after irradiation. Because of this delay in counting, we were unable to see all the peaks in the different spectra. In the case that a certain gamma ray peak was not seen, an upper bound was given as previously discussed. In the case that the upper bound overestimated (gave a percentage composition greater than 50\%), the composition was left as undetermined, as we feel it is extremely unrealistic and unhelpful to state such a high upper bound. 

\begin{table}[h]
\centering
\caption{ This table shows various elemental composition percentages of metals for the fish NAA.}
\resizebox{\columnwidth}{!}{
\begin{tabular}{ | l | l | l | l | l | l | l | l | l | }
\hline
	Sample ID & \% Se & \% Hg & \% As & \% Cs & \% Rb & \% Zn & \% Na & \% Br \\ \hline
	Halibut - 6/6/15 & 0.065 $\pm$ .003 & 0.00600 $\pm$ 0.00008 & . $\pm$ .008 & 0.00141 $\pm$ 0.00001 & 0.008 $\pm$ 0.001  & .046 $\pm$ 0.002 & 9.5 $\pm$ .3 & .378 $\pm$ .004  \\ \hline
	Wild King Salmon - 4/2/15 & 0.0188 $\pm$ .0001 & 0.001430 $\pm$ 0.000008  & Undetermined & 0.00235 $\pm$ 0.00002  & .011 $\pm$ 0.001 & .052 $\pm$ 0.002 & Undetermined & Undetermined \\ \hline
\end{tabular}



}
\end{table}




\subsection{Seaweed Long Run Results}

\begin{table}[h]
\centering
\caption{ This table shows various elemental composition percentages of metals for the seaweed NAA.}

\begin{tabular}{ | l | l | l | l | l | }
\hline
	Sample ID & \% As & \% Rb & \% Zn \\ \hline
	Sea Palm & 0.013 $\pm$ .001 & 0.0013 $\pm$ .0001 & 0.0017 $\pm$ 0.0002 \\ \hline
	Wakame & $<$ 0.0020 & 0.0011 $\pm$ .0001 & 0.0021 $\pm$ 0.0002 \\ \hline
\end{tabular}



\end{table}


\subsection{Seashell Long Run Results}



\begin{table}[htb!]
\centering
\caption{ This table shows various elemental composition percentages of metals for the seashell NAA.}
\resizebox{\columnwidth}{!}{
\begin{tabular}{ | l | l | l | l | l | l | l | l | l | }
\hline
	Sample ID & \% Ba & \% Cr & \% Zr & \% Fe & \% Sc  & \% Br \\ \hline
	Seashell 2 & 2.4 $\pm$ 0.1 & 0.21 $\pm$ 0.06 & 0.030 $\pm$ 0.001 & 1.8 $\pm$ 0.1 & 0.0020 $\pm$ 0.0003 & 0.139 $\pm$ 0.001 \\ \hline
	Seashell 3 & Undetermined & Undetermined & Undetermined & 3.0 $\pm$ 0.2 & 0.00062 $\pm$ 0.0004 & 0.139 $\pm$ 0.002  \\ \hline
\end{tabular}



}
\end{table}

\pagebreak

\section{Conclusion}
Neutron activation analysis provides us with a useful tool in determining the composition of various samples. Using the standard pottery, we were able to thoroughly analyze the initial kelp NAA data, in addition to some samples in the second batch of NAA carried out at MNRC. As shown in this report, kelp is primarily composed of potassium and sodium, with trace amounts of scandium, rubidium, iron, and zinc. Our primary analysis of fish shows sodium, bromine, arsenic, selenium, rubidium, cesium, and mercury. While the former five elements are expected to be found in sea life, the latter two, cesium and mercury, may be worrisome. Cesium is a product of nuclear fission, and this can mean the fish samples in question ingested cesium from the Fukushima fallout. Mercury is also a troublesome element to find in sea life as it a product of pollution of the ocean. The seaweed samples produced similar results to the initial kelp samples, as expected, and the seashells provided a fair amount of trace metals.


Unfortunately, due to logistical problems with acquiring the second batch of samples, we were unable to thoroughly analyze the second batch of samples. The results shared in this report are preliminary findings with regard to the second batch of samples. In the case of the short run analysis, we hope to receive efficiency data from MNRC to calculate compositions from the short run counting.


Future work will involve analyzing the short run data and the second batch of samples in more depth in preparation for publication to the RadWatch website. The next section provides a rough draft of the article we hope to post to the website after the completion of the data analysis. Additionally, next steps for this project may be to analyze new kelp samples from the same locations, to observe any elemental composition differences over time. 

\pagebreak


\section{RadWatch Article}

\textbf{What is NAA?}

Neutron Activation Analysis (NAA) involves irradiating test samples with a high neutron flux in order to activate the isotopes present in the sample. These isotopes decay, emitting characteristic gamma rays that allow for the identification the isotopes that were originally in the sample. Based on the half-life of the isotope, its branching ratio at a particular gamma energy, and the counts registered on the detector, we can also determine the amount of isotope in the test sample.

\textbf{Why do we use NAA?}


NAA provides advantages that other techniques do not. First, NAA is relatively unobtrusive, and will not cause significant damage to the test sample. As neutrons can penetrate deeply into a material, NAA can be used on bulk samples. These samples do not need the careful preparation that is needed for other techniques, such as chemical ones, to determine elemental composition.

\textbf{Historical Examples?}


Uses of NAA span from forensic science where it can be used to analyze crime scene materials such as bullets, metals, glass, etc., to archeology where it helps in determining the sources and age of certain pottery discovered. One example of the use of NAA deals with a painting that was thought to have been drawn by the artist Rembrandt. The Berlin Hahn-Meitner Institute for Atomic Research subjected the painting to 70 hours of neutron irradiation in order to activate the pigment neutrons and compare their behavior with those of genuine Rembrandt’s work. Paint and pigment analysis later showed that this Rembrandt painting was not painted by him, but by one of his students. 


\textbf{What we did and why?}


Our team aimed to determine the elemental composition of kelp, fish, and other sea life along the west coast of the Americas, particularly in the San Francisco Bay and Long Beach areas, although the samples used were from Alaska to Chile. In particular, we wanted to see if there are any major differences in isotopes found in samples from different parts of the West Coast, and whether the isotopes found in the different samples presented a toxicity or radiological health hazard to the public. Specifically, we collected samples of seashells near the San Francisco Bay which were later irradiated at the McClellan Nuclear Research Center. The activated samples were then analyzed in order to determine their elemental compositions. Samples of fish and seaweed were also counted. 

\textbf{Data and Results?}

This section will include the various tables of results presented in this report. Additionally, we will include figures as a means of visualizing the data. An example figure is shown in Figure 11.

\begin{figure}[htb!]
\centering
\fbox{\includegraphics[scale=0.4]{ExampleChart}}
\caption{This figure shows the elemental composition of potassium in the original 15 long run kelp samples. Samples without error bars show an upper bound on the potassium concentration. The raw data can be found in Table 1.}
\end{figure}

\pagebreak

\textbf{Future Work?}

Future work that can be done for this project would be to survey other areas that are highly populated and near large harbors. Areas such as New York, Boston would be good locations to perform NAA on sea samples to determine if there any harmful substance in their waters. 

\pagebreak

\section{Acknowledgements}

We would like to thank Keenan Thomas and Professor Kai Vetter for helping us determine the scope of this project, and for the assistance they give us during the course of completing this project. Furthurmore, we would like to thank Wesley Frey and the rest of the MNRC staff for assisting us with the irradiation. Finally, we would like to thank Max Fratoni for his guidance and assistance during the NE170 course and the completion of this project.


\end{document}




